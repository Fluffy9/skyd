\documentclass[twocolumn]{article}
\usepackage{amsmath}

\begin{document}
\frenchspacing

\title{Skynet: An Application Layer for Decentralized Storage}

\author{
{\rm David Vorick}\\
Nebulous\\
}

\maketitle

\subsection*{Abstract}
We introduce Skynet, a next generation platform for building Internet
applications. Skynet builds on the advantages of the cloud, allowing developers
to launch rich, data heavy applications without needing to run any of their own
infrastructure. Like the cloud, applications enjoy near perfect uptime, high
performance, and almost unlimited scalability.

Unlike the cloud, applications live in a global dataspace. When developers
launch a new application, they can build upon all of the content and knowledge
that users have generated across all applications, rather than needing to
bootstrap an ecosystem and overcome network effects.

Also unlike the cloud, Skynet is user driven rather than developer driven. Users
control not only the data flow of an application, but also the upgrade path. A
favored version of an application can live forever, even if the developers have
moved on or released unwanted upgrades.

\section{Introduction}
Many of the early pioneers of the Internet would say that it has failed to
deliver on its early potential. There is no question that the Interent has
dramatically changed society and enabled an enormous number of positive things
that would never have been possible otherwise. But the original vision for the
Internet emphasized ideals around the freedom of information. The Interent was
supposed to be a way for humanity to collaboratively build a collective
knowledgebase that benefits everyone.

Instead, the Internet today more closely resembles the feudalism of the middle
ages. The internet is broken into largely independent fiefs, where content is
generated by peasant-like users, and value accrues to the lord-like platforms.
Though the users put in the work, ownership of the result is generally legally
assigned to the platform.

The importance of network effects puts these platforms in an enormous position
of power over users. Platforms require users comply with strict and frequently
abusive terms of service, creating a situation where content creators and
consumers alike strongly dislike the platforms they use, yet due to a dependence
on network effects find themsevles without any viable alternatives.

Skynet has been built to correct this issue by making data globally available,
rather than locked to a single platform. This allows network effects to extend
beyond a single platform, enabling users to switch to new or alternative
platforms, without losing any of the network effects that they enjoyed on the
original.

Skynet transforms the Internet from a fragmented collection of closed platforms
into a collective body of human knowledge that is accessible to everyone. Skynet
brings the Internet back to its original vision.

\section{Technological Foundation}
Skynet is built on top of the Sia \cite{sia} network. Sia is a decentralized
storage network which allows users to upload and share data similar to using the
cloud, without requiring the users to sign up for a centralized cloud provider.
Unlike many peer-to-peer storage systems, Sia does not require users to keep
machines online or seed data back to the network. Sia instead leverages a
financial economy where users pay service providers through a decentralized
blockchain to keep their data online.

Sia has a high performance architecture, which allows data to be uploaded and
downloaded at speeds comparable to the centralized web. This is important for
application builders, as it means they do not need to compromise user experience
when choosing to build decentralized applications.

\section{The IGDL}
The IGDL, short for 'Immutable Global Data Layer', is an IPFS \cite{ipfs}
inspired storage layer for immutable data. Data uploaded to the IGDL is
referenced by its Merkle root, meaning that anyone who knows the Merkle root of
a piece of data can download that data from the IGDL. If the same file is
uploaded multiple times, it will have the same hash every time.

The IGDL improves upon IPFS by having Sia host the content. This means that once
data is uploaded, users do not need to stay online or use a pinning service to
ensure that the data remains available. The Sia network handles uptime,
performance and scalability for the user.

Application code can go onto the IGDL. As an example, there is a sudoku solver
\cite{c-sudoku} that exists in the IGDL and can be run in the web browser. These
applications can themselves upload to and download from the IGDL using the
Skynet API. Any node that is capable of fetching content from the IGDL also
provides access to an API for applications running from the IGDL. Skybin
\cite{c-skybin} is a simple example of a pure-Skynet app which uploads new data
to the IGDL. Skybin takes a piece of text provided by a user and uploads that
text to the IGDL, presenting a link to the user that they can share with
friends.

Data uploaded to the IGDL can take the form of a single file, or it can take the
form of an entire file tree. More complex applications often require loading
assets from a folder structure. The Uniswap frontend \cite{c-uniswap} is an
example of an application that uses an entire filesystem tree. Instasky
\cite{c-instasky} is application in the IGDL that you can use to view the
directory structure of IGDL links. You can use InstaSky to see how the Uniswap
frontend was built \cite{c-instasky-uniswap}.

The content format of the IGDL includes JSON metadata. Some fields are Skynet
specific, and allow uploaders to specify things such as wallet addresses for
donations, or configuration settings such as which file to use as the index file
for an application. Uploaders also have a place to put arbitrary fields into the
metadata, allowing some files to provide metadata that may be application
specific and not explicity defined in the Skynet specification.

\section{User Filesystems}
The IGDL is great for applications and content, but the immutability of the data
limits its overall usefulness as a foundation for stateful applications. We can
overcome these limitations by giving applications access to a persistent
filesystem that the user keeps on the Sia network.

Each application gets access to its own folder. Within that folder, the
application can create and modify files which will persist between uses.
Skymarks \cite{c-skymark} is an example application on the IGDL which makes use
of its application folder to store bookmarks for a user. When the application
loads, it queries the user's filesystem to see if any bookmarks have previously
been saved by the user, and can display those bookmarks to the user.

The user-oriented data model means that application developers do not need to
worry about storage or bandwidth themselves. Users manage the storage,
bandwidth, and associated costs when using the application. Because the data is
under user control, application developers also do not need to worry about
liabilities related to data breaches and data regulation. Developers do not need
to handle or store user data when developing Skynet applications.

Within the application folder, there are two sub-folders that indicate how the
data interacts with other users and applications. There is the 'private' folder
and the 'public' folder. Data in the private folder is only visible to the one
application. Other applications are not allowed to view this folder unless the
user explicitly gives that application access rights. Data in the 'public'
folder can be viewed by anyone on Skynet who knows the user's identity. This
includes other applications that the user runs.

The Skyblogger \cite{c-skyblog} application makes effective use of all of these
folders. A blogger's draft blog posts are placed in to the private folder of the
application, and their completed blog posts get placed into the public folder.
Readers and followers of the blogger will scan the public folder of the blogger
each time they use a blog reader, which allows them to see new posts.

When building more rich applications, the public folder can be used to store
data such as likes, follows, and comments. Skynet uses the public folder system
to store data in place of the traditional centralized database model.

\section{User Identity}
Users can create identities which can be shared with friends or published
broadly. These identities are used to access the public application folders of a
user. If desired, a user can create many identities and selectively share each
identity with friends. This allows users to cleanly separate roles in their
life, for example a user could have a work identity and a personal identity.

An identity itself consists of an encryption key and a list of file contracts on
Sia. The list of file contracts are used to look up the public mutable data that
gets published under the identity. Rather than being identified by a hash, data
is placed at a specific offset within each contract, which is shared as a part
of the identity. This data is encrypted using the encryption key, meaning that
only people who know an identity can view the data.

The full list of file contracts is often several kilobytes, quite a bit longer
than the typical 46 byte Skylink. Identities can be uploaded to the IGDL so that
they can be shared more easily, however for the most part users will generally
not see identity links directly. Instead, most applications will have 'follow'
and 'send friend request' buttons - for the most part, identities on Skynet will
feel very similar to identities on traditional social networks, the internal
cryptography and distributed systems work does not need to be apparent to the
user.

\section{Contributor Incentives}
Skynet features an incentives system which allows people to get paid for
contributing to Skynet. Contributors are broadly defined, as there are many
roles that add value to the Skynet ecosystem.

One of the most important roles is the developer role. For developers, Skynet
features a fee extraction endpoint in the API. As users run the code that a
developer has written, that code can request fees from the user's Skynet node,
providing both an address and an amount to be paid. This endpoint is accessible
to libraries as well as full applications, meaning that a useful library which
gets included in many applications can be well compensated even if the author
never makes an application of their own.

A second key role in the Sia ecosystem is content creators. Content creators can
get compensated by filling out the metadata fields for earning fees in the
content that they upload to the IGDL. Each time this content is used, a fee will
be requested and sent to the addresses listed within the content.

Importantly, content creators can get compensated for their content under this
model even if that content is used without explicity requesting permission. For
example, if a photographer uploads a photograph to a social media app, and then
that photograph gets used by a popular blogger, the photographer will be
automatically compensated every time the blog post is read. The blogger does not
need to take any steps to contact the photographer or get explicit permission,
they only need to link to the original file in the IGDL.

This low friction compensation model also applies to developers who write
libraries for other applications to use. Just by importing the code, an
application is ensuring that the authors are getting rewarded for building a
useful library. No business transaction needs to occur.

Developers can leverage this same fee model to compensate users within the
ecosystem of an application. One example might be a content curator who creates
music playlists for other people to enjoy. A developer could potentially design
an application so that every time a user benefits from a playlist made by a
curator, that curator receives a fee from the user.

\section{Accessing and Using Skynet}
Accessing Skynet in a fully decentralized way requires running a Sia full node.
The resource requirements of this are fairly tolerable, the only hard
requirements are an SSD with at least 32 GB of free space and 4 GB of RAM.
Hardware that meets this requirement can be bought for less than \$100. From
there, a typical user is going to need Siacoins and can expect to spend \$5 to
\$10 per month on storage, bandwidth, and network fees. Heavy users can expect
to spend more - Sia is metered, similar to many utilities.

Most users are not expected to run a full node in the short term. Instead, we
expect that most users will be accessing Skynet through a third party service
called a portal. Using a portal for Skynet is similar to using a custodial
exchange for Bitcoin \cite{bitcoin} or Ethereum \cite{ethereum}. The portal is a
trusted entity that controls all of the user's Skynet files and identities.

Portals are in a position to abstract away all of the novel elements of
accessing Skynet. Users can pay with credit cards instead of cryptocurrency, and
users can access Skynet through a traditional web browser, they do not need any
special software or browser extensions.

Infrastructure costs on Skynet are low enough that portals can affordably offer
generous free tiers for users. Most users will not need to pay for Skynet until
after many hours of usage. Even then, portals can maintain a free tier using
stratiegies such as advertising. A user watching a video advertisement every
10-20 minutes should sustain the portal's costs indefinitely.

As of writing, more than 10 portals are available to the public. One is operated
by the commercial entity behind Sia and Skynet, and the other 9 are all
community operated.

\section{User Spending Controls}

\section{Censorship Resistance}
Sia, the underlying storage platform, is a fully peer to peer network with no
central element of control. When data is uploaded to Sia, it gets spread across
many nodes. If some of those nodes go offline, the data is automatically
repaired onto other nodes. At the network level, the only way to effectively
remove data from Sia is to contact all hosts who posses data and have them all
remove that data at once.

This assumes that all of the locations of a piece of data are known. When data
is shared publicly on Sia, often a portion of the redundancy is withheld, so
that the original uploader can repair the file even if hosts are coordinating to
attack the file.

At the user level, a user can be contacted to remove a file. If the user
complies, all copies of the file managed by that user will be removed from the
network. For files in the IGDL, often multiple users will be managing the same
file. At the user level, removing a file from the IGDL requires coordinating
with all users who are managing the file. So long as a single user continues to
repair the file, the file will remain available globally.

Portals are centralized entities that have the ability to blacklist files. The
portals cannot remove the files from the Sia network, however they can deny
access to all of their users. Users however can either run their own full node
or find another portal to circumvent blocks put in place by portals.

\section{Privacy}
Sia has sophisticated encryption that successfully disguises most types of
network activity. Someone looking at the network traffic coming from a Sia node
will be able to tell how much data is being transferred, and little else. All
communications are encrypted, and all messages are padded to a full packet's
worth of data. Generally speaking, an Internet Service Provider (ISP) that is
watching user traffic is going to be unable to tell what type of browsing a user
is doing.

With traditional Internet traffic, the ISP learns a significant amount of
information about the user's browsing habits. The ISP is typically able to see
all of the websites that a user is visting, and even though SSL prevents the ISP
from learning the exact requests, knowing which applications the user prefers
alone gives the ISPs substantial information about the user.

If users are using the Sia network directly, the hosts have the ability to tell
what the user's activities are. The user's requests for files, applications, and
friends data are all visible. This puts the hosts in a similar position to the
ISPs for traditional Internet apps - hosts cannot see the actual data being
accessed, but can learn a significant amount from the metadata.

This can be resolved by using a VPN. If a user is behind a VPN, the hosts will
be able to see what type of traffic is happening, but will have a much harder
time identifying which traffic belongs to which user, especially if many users
are using the same VPN. The VPN will be in much the same position as the
Internet providers - the VPN can tell that a user is using Skynet, but has no
idea which applications the user is accessing. Fully unmasking the user would
require the VPN and the hosts to collaborate.

Eventually, the Sia network plans to allow users to bounce traffic between hosts
before making a final request. This would substantially reduce the ability of a
host to correlate traffic from a user, as every request would be completely
disconnected from the others. This could be done without introducing much
latency, as the hosts typically have good connections to eachother and get paid
for their bandwidth.

\section{Decentralization}
Both Skynet and the Sia network are fully peer-to-peer systems. No single
individual or political power runs infrastructure which is critical to the
operation of the network - any participant, or indeed even large subsets of
participants, can leave the network at any time without disrupting the
effectiveness or user experience of the two networks.

Neither network features any form of explicit governance. All software is fully
open source, and all upgrades require users to opt-in. Incompatible upgrades
will always result in a network split, and users always have the option to
reject the upgrade.

\section{Example: Skynet Twitter}
This section explains one method that someone might use to build a Skynet based
alternative to Twitter. We will call this app Skatter. To keep things simple, we
say that Skatter needs to fulfill the following roles:

\begin{itemize}
	\item Users can publish statements of up to 140 characters in size
	\item Users can discover and follow other users
	\item Users can like and re-tweet messages
	\item Developers and users alike are compensated financially
\end{itemize}

When a user gets started with skatter, they will need to create an account. When
they click 'Create Account' in the application, they grant Skatter permission to
create files and folders within the 'skatter' application directory. Upon
account creation, two files will be created. The first file logs a user's
interactions with statements. The second file is a list of all the people that
the user follows.

Three types of activity are supported:

\begin{itemize}
	\item Creating a new statement
	\item Liking a statement
	\item Skattering a statement (re-tweeting)
\end{itemize}

Statements themselves are not placed into the log, but instead placed onto the
IGDL. When a statement is placed onto the IGDL, two fees are attached. One that
pays into the account of the person who created the statement, and one that pays
into the account of the Skatter developers. Every time the statement is viewed,
both parties will receive some small compensation (tuned to cost users about 10
cents per hour of browsing). Some metadata will also be added to the file which
identifies the user who originally created the statement.

The log in the user's public folder contains links to statements. The links are
paired with one of three notes: 'created', 'liked', and 'skattered'. These notes
indicate what type of interaction the user had with the linked statement.

Upon startup, Skatter will look in the user's folder and check the list of
people that the user follows. Skatter will then scan all those people's public
skatter folders and read their interaction logs. This allows the Skatter app to
build a feed for the user that contains all of the recent statements, likes, and
skatters made by the people they follow.

Because of the skatter mechanism, there may be statements in the user's feed
that were made by people that the user does not follow. If the user likes the
statements, they can choose to follow the statement creator. From the user's
perspective, they just click a 'follow' button. In the background, the Skatter
app will look at the metadata of the statement to find the identity of the
author. That identiy will then be added to the file which contains the list of
all people that the user follows.

Because all of the statements and identities themselves are published on the
IGDL, some other set of developers can make an application which makes use of
the content published via Skatter. This could even be a Skatter competitor. The
content in the IGDL specifies fees that go to the Skatter creators, meaning that
even if another platform is using Skatter content, the Skatter devs are being
compensated. This allows for much more rapid and permissionless innovation
overall, while still protecting the income of the original builders.

\section{Conclusion}

\onecolumn
\begin{thebibliography}{9}

\bibitem{sia}
	Sia: Simple Decentralized Storage \newline
	https://sia.tech/sia.pdf \newline
	sia://XABvi7JtJbQSMAcDwnUnmp2FKDPjg8\_tTTFP4BwMSxVdEg

\bibitem{ipfs}
	IPFS - Content Addressed, Versioned, P2P File System (DRAFT 3) \newline
	https://ipfs.io/ipfs/QmR7GSQM93Cx5eAg6a6yRzNde1FQv7uL6X1o4k7zrJa3LX/ipfs.draft3.pdf \newline
	sia://XAGsqR8EMD1eIP17NQ4h4\_6UNyE66oyTVFfbfedcNyoF7g

\bibitem{c-sudoku}
	Sudoku Application \newline
	https://siasky.net/LAAW-FGPHXIjOo31FdUQIho1wxBt20rWaG5Gy-zARLz2LA \newline
	sia://LAAW-FGPHXIjOo31FdUQIho1wxBt20rWaG5Gy-zARLz2LA

\bibitem{c-skybin}
	Skybin Application \newline
	https://siasky.net/CAAVU14pB9GRIqCrejD7rlS27HltGGiiCLICzmrBV0wVtA \newline
	sia://siasky.net/CAAVU14pB9GRIqCrejD7rlS27HltGGiiCLICzmrBV0wVtA

\bibitem{c-uniswap}
	Uniswap Frontend \newline
	https://siasky.net/EAC5HJr5Pu086EAZG4fP\_r6Pnd7Ft366vt6t2AnjkoFb9Q/index.html \newline
	sia://EAC5HJr5Pu086EAZG4fP\_r6Pnd7Ft366vt6t2AnjkoFb9Q/index.html

\end{thebibliography}

\end{document}
